% -*- Mode: LaTeX; TeX-PDF-mode: t -*-
\documentclass{article}
\usepackage{amsmath}
\newcommand\paren[1]{\left( {#1} \right)}
\newcommand\set[1]{\left\{ {#1} \right\}}
\newcommand\sched{\mapsto^S}
\newcommand\wake[1]{\mapsto^W_{#1}}
\newcommand\vcpu[1]{\paren{#1}}

\begin{document}

\section{Scheduling}

\paragraph{Terminology} $\tau$ denotes a task, $Q$ denotes the quantum
(in time units), $n$ denotes remaining time, $\sched$ denotes {\tt
  schedule}, and $\wake{\tau}$ denotes {\tt wakeup} on task $\tau$.
VCPUs are represented as a tuple $\vcpu{\tau, n, \tau, \ldots}$
containing the current task, the remaining quantum, and the tasks in
the runqueue.  For convenience, assume that the runqueue entries are
filled with the idle task $\tau_0$ whenever needed.

\subsection{Second level scheduling}
\begin{flalign}
  \vcpu{\tau, 0, \tau', \tau'', \ldots} &\sched \vcpu{\tau', Q, \tau'', \ldots} \label{vcpu:s1}\\
  \vcpu{\tau, n, \ldots, \tau, \ldots} &\sched \vcpu{\tau, n, \ldots}\quad\text{where }n > 0 \label{vcpu:s2}\\
  \vcpu{\tau, n, \tau', \ldots} &\sched \vcpu{\tau', n, \ldots}\quad\text{where }n > 0, \tau\neq\tau' \label{vcpu:s3}\\
  \vcpu{\tau, n, \tau', \ldots} &\wake{\tau''} \vcpu{\tau, n, \tau', \ldots, \tau''}\quad\text{where }\tau''\not\in\set{\tau',\ldots}\label{vcpu:w1} \\
  \vcpu{\tau, n, \ldots, \tau'', \ldots} &\wake{\tau''} \vcpu{\tau, n, \ldots, \tau'', \ldots} \label{vcpu:w2}
\end{flalign}

Transition~\ref{vcpu:s1} occurs when the quantum expires.
Transition~\ref{vcpu:s2} occurs when the quantum has not expired and
the current task also appears on the runqueue.
Transition~\ref{vcpu:s3} is the remaining case for when the quantum
has not expired but the task is not runnable.
Transition~\ref{vcpu:w1} shows how tasks are added to the runqueue.
Transition~\ref{vcpu:w2} indicates idempotency of {\tt wakeup}.

\end{document}
